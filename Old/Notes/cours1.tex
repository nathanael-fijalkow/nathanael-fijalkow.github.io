\documentclass{article}
\usepackage[utf8]{inputenc}
\usepackage[french]{babel}
\usepackage[T1]{fontenc}
\usepackage{graphicx}
\usepackage{hyperref}
\usepackage{textcomp}
\usepackage[top=2cm, bottom=2cm]{geometry}
\usepackage{amsmath}
\usepackage{amssymb}
\usepackage{stmaryrd}
\usepackage{cancel}
\usepackage{tikz}
\usetikzlibrary{shapes,snakes}

\usepackage{array}

%mise en page document
\usepackage{fancyhdr}
\pagestyle{fancy}

\fancyhf{}
\renewcommand{\headrulewidth}[0]{0pt}
\fancyfoot{}
\renewcommand{\footrulewidth}[0]{0.00mm}
\fancyfoot[L]{4TIN919U-Jeux, synthèse et contrôle}
\fancyfoot[R]{\thepage}


\begin{document}

%-------title---------------------------------------
\begin{center}
\Large{\textbf{Jeux, synthèse et contrôle}}
\vskip2em
Notes par Sarah Larroze
\end{center}

\section*{Introduction}
Ce cours s'intéresse à l'apprentissage (Machine learning). Cette notion est liée
à la complexité, à la logique, à l'algorithmique et à la statistique. Le principe
de l'apprentissage est de classifier des valeurs d'un domaine $X$ à l'aide d'un
label set $Y$. La plupart du temps, $Y=\left\{0,1\right\}$. Il existe différents
types d'apprentissage :

    \paragraph{L'apprentissage supervisé}
        \begin{tabular}{l c l}
         Input & : & $\begin{pmatrix}
                         x_1 \\
                         y_1 \\
                     \end{pmatrix}, \dots,
                         \begin{pmatrix}
                         x_n \\
                         y_n \\
                         \end{pmatrix}$ \\
        Output & : & $h : X \rightarrow Y$\\
        \end{tabular}

    On obtient une classification si $Y=\left\{0,1\right\}$ avec 0 qui étiquette
    les $x\in X$ négatifs et 1 qui étiquette les $x\in X$ positifs. Le but de
    l'apprentissage supervisé est de généraliser sur des entrées inconnues ce qui
    a été appris à partir des $\begin{pmatrix}
                    x_1 \\
                    y_1 \\
                \end{pmatrix}, \dots,
                    \begin{pmatrix}
                    x_n \\
                    y_n \\
                    \end{pmatrix}$.

    \paragraph{L'apprentissage non supervisé}
        \begin{tabular}{l c l}
         Input & : & $\begin{pmatrix}
                         x_1, \dots, x_n \\
                     \end{pmatrix}$\\
        Output & : & une partition de $X$ en $k$ classes avec $k \in \mathbb{N}$\\
        \end{tabular}


        Le but de l'apprentissage non supervisé est de partitionner un ensemble
        d'entrées en fonction de similarités ou différences caractérisant cet
        ensemble.

    \paragraph{Reinforcement learning (Apprentissage par récompense), noté RL}~\\

        \begin{center}
        \begin{tikzpicture}
            \begin{scope}[shape=ellipse,minimum size=1cm]
                \tikzstyle{every node}=[draw]
                \node (q_1) at (0,0){Agent};
                \node (q_2) at (4,0) {Environnement};
                \end{scope}
                \draw[->] (q_1) .. controls +(up:1cm) and +(up:1cm) .. node[above] {$a\in Actions$} (q_2);
                \draw[->] (q_2) .. controls +(down:1cm) and +(down:1cm) .. node[below] {$\mathit{r\acute{e}compense}\in \mathbb{R}$} (q_1);
        \end{tikzpicture}
    \end{center}

        L'agent doit apprendre les actions à prendre pour maximiser la récompense
        en fonction de l'environnement.

    \paragraph{Online learning (Apprentissage au fur et à mesure)} Cet apprentissage
    correspond à une succession de phases d'apprentissage. Chaque phase d'apprentissage est constituée de la manière suivante : \\
    \begin{tabular}{l c l}
     Input & : & $\begin{pmatrix}
                     x_k \\
                     y_k \\
                     \end{pmatrix}, k\in \mathbb{N}$ \\
    Output & : & $h_k : X \rightarrow Y$\\
    \end{tabular}

    La fonction $h_k$ retournée pour la phase $k$ doit être meilleure que $h_{k-1}$.

    \section{Apprentissage supervisé}
    On rappelle que l'apprentissage supervisé est de la forme \begin{tabular}{l c l}
     Input & : & $\begin{pmatrix}
                     x_1 \\
                     y_1 \\
                 \end{pmatrix}, \dots,
                     \begin{pmatrix}
                     x_n \\
                     y_n \\
                     \end{pmatrix}$ \\
    Output & : & $h : X \rightarrow Y$\\
    \end{tabular}.

    $S$ est appelé le "sample". Soit $\mathcal{H}\subseteq \left\{f : X \rightarrow Y\right\}$ l'ensemble des fonctions de retour acceptées pour l'apprentissage. On considère alors le
    problème suivant:

    \begin{tabular}{l c l}
     Input & : & $S = \left\{ (x_i,y_i) \backslash i\in \llbracket 1, \lvert S \rvert  \rrbracket \right\}$ \\
    Output & : & Trouver $h\in \mathcal{H}$ tq $h\models S$ (i.e $\forall i, h(x_i)=y_i$)\\
    \end{tabular}

    \paragraph{Définition} Un algorithme apprend $\cal{H}$ si pour tout $S$,
    l'algorithme retourne $h\models S$ avec $h\in \cal{H}$ ou "il n'existe pas
    de $h\in \mathcal{H},  h\models S$". De plus, il est efficace s'il est polynomial en la dimension des $x_i$ et le nombre de points et on note
    $poly(dim(x_i),\lvert S \rvert)$.

    \subsection{Formules conjonctives}
    Dans cette partie, $X = \left\{0,1 \right\}^n, n\in \mathbb{N}$, $Y = \left\{0,1 \right\}$ et $\mathcal{H} = \left\{\varphi = \bigwedge_{p\in X_1} p \wedge \bigwedge_{p\in X_2} \neg p \right\}$.

    Par exemple, pour $n=2$, $\mathcal{H} = \left\{p_1, p_2, p_1 \wedge p_2,
    p_1 \wedge \neg p_2, \neg p_1 \wedge p_2, \neg p_1 \wedge \neg p_2 \right\}$.

    On considère le sample $S = \left\{(x_1,y_1),\dots,(x_5,y_5) \right\}$ avec
    $dim(x_i)=5$ suivant :

    \begin{center}
        \begin{tabular}{c c c c c c c c c}
        & $p_1$ & $p_2$ & $p_3$ & $p_4$ & $p_5$ &&&\\
        $x_1$ & 0 & 1 & 0 & 1 & 1 && 1 & $y_1$ \\
        $x_2$ & 0 & 1 & 1 & 1 & 0 && 1 & $y_2$ \\
        $x_3$ & 0 & 1 & 1 & 0 & 1 && 0 & $y_3$ \\
        $x_4$ & 1 & 0 & 1 & 1 & 0 && 0 & $y_4$ \\
        $x_5$ & 0 & 0 & 0 & 1 & 1 && 0 & $y_5$ \\
        \end{tabular}
    \end{center}

    On veut alors obtenir $\varphi$ tq $x_1 \models \varphi$ et $x_2 \models \varphi$ et $x_3 \not\models \varphi$ et et $x_4 \not\models \varphi$ et $x_5 \not\models \varphi$. L'algorithme utilisé pour trouver une telle formule
    $\varphi$ est de considérer la conjonction contenant tous les $p_i$ et
    leur négation. On supprime au fur et à mesure les valeurs des $p_i$ qui
    empêchent de satisfaire le sample. \`A la fin, si $\varphi \models S$ on
    retourne $\varphi$, sinon on retourne "il n'existe pas de $\varphi$".
    \\

    Appliquons cet algorithme que le sample $S$ précédent: \\
    Initialement, $\varphi = p_1 \wedge \neg p_1 \wedge p_2 \wedge \neg p_2 \wedge p_3 \wedge \neg p_3 \wedge p_4 \wedge \neg p_4 \wedge p_5 \wedge \neg p_5$\\
    $x_1 \models \varphi$ : $\varphi = \cancel{p_1} \wedge \neg p_1 \wedge p_2 \wedge \cancel{\neg p_2} \wedge \cancel{p_3} \wedge \neg p_3 \wedge p_4 \wedge \cancel{\neg p_4} \wedge p_5 \wedge \cancel{\neg p_5}$\\
    $x_2 \models \varphi$ : $\varphi = \cancel{p_1} \wedge \neg p_1 \wedge p_2 \wedge \cancel{\neg p_2} \wedge \cancel{p_3} \wedge \cancel{\neg p_3} \wedge p_4 \wedge \cancel{\neg p_4} \cancel{\wedge p_5} \wedge \cancel{\neg p_5}$\\
    On vérifie ensuite que : $x_3 \not\models \varphi$, $x_4 \not\models \varphi$,
    $x_5 \not\models \varphi$. C'est bien le cas ici donc on retourne $\varphi = \neg p_1 \wedge p_2 \wedge p_4$.\\

    On note $n=dim(x_i)$, cet algorithme s'effectue en $O(n\lvert S \rvert)$ donc
    $\mathcal{H}$ est apprenable efficacement.

    \subsection{DNF (Disjonctive normal form)}
    On va à présent considérer le cas où $\mathcal{H}$ est l'ensemble des formules sous forme normale disjonctive. Par définition, $\varphi$ est sous forme
    normale disjonctive si $\varphi = \bigvee_{i=1}^k C_i, C_i = \bigwedge_{p\in X_1} p \wedge \bigwedge_{p\in X_2} \neg p$. Dans cette partie, $X = \left\{0,1 \right\}^n, n\in \mathbb{N}$ et $Y = \left\{0,1 \right\}$.

    On reprend l'exemple précédent et on cherche à nouveau à savoir s'il est possible d'obtenir une formule $\varphi$ tq $\varphi \models S$.

    Pour cela, il suffit de prendre :
    \begin{equation}
        \varphi = \underbrace{(\neg p_1 \wedge p_2 \wedge \neg p_3 \wedge p_4 \wedge p_5)}_{x_1 \models \varphi}
                \vee \underbrace{(\neg p_1 \wedge p_2 \wedge p_3 \wedge p_4 \wedge \neg p_5)}_{x_2 \models \varphi}
    \end{equation}

    On peut donc toujours construire $\varphi$ pour qu'elle ne satisfasse que
    les entrées à vérifier. Il faut ensuite vérifier qu'il n'y a pas d'incohérence
    dans le sample (i.e $\not \exists x_i, x_j, i \neq j$ avec $y_i = y_j$).
    On remarque alors que $\mathcal{H}$ est trivialement apprenable. En revanche,
    le consistency model ne prend pas en compte la généralisation car il se contente d'apprendre les exemples positifs et ne met donc pas certaines
    propriétés en avant.

    \subsection{3-DNF}
    Dans cette section, $\mathcal{H}_3 = {\varphi = C_1 \vee C_2 \vee C_3}$. On
    s'intéresse au problème de la 3-coloration qui se résume au problème suivant:
    \\

    \begin{tabular}{ccc}
    Input & : & $G = (V,E)$ un graphe non-orienté \\
    Output & : & $\exists c : V \rightarrow \left\{ R,G,B \right\} tq~ \forall e=(u,v)\in E, c(u) \neq c(v)$ \\
    \end{tabular}
    \\

    \'Etant donné un graphe $G$, on cherche à construire un sample
    $S = \left\{(x_i, y_i)\right\}$ tq $G$ est 3-coloriable ssi $\exists \varphi \in \mathcal{H}_3 \backslash \varphi \models S$.

    L'exemple suivant présente un graphe 3-coloriable (voir couleur des sommets)
    et le sample le représentant. La partie supérieure du sample représente les
    sommets du graphes et la partie inférieure représente ses arêtes.

    % faire graphe d'exemple
    \begin{minipage}[c]{0.45\linewidth}
            \begin{tikzpicture}[scale=1.5]
                \begin{scope}[shape=circle]
                    \tikzstyle{every node}=[draw]
                    \node[fill=red!50!] (1) at (0,0){1};
                    \node[fill=green!50!] (2) at (2,0){2};
                    \node[fill=blue!50!] (3) at (2,-2){3};
                    \node[fill=green!50!] (4) at (0,-2){4};
                    \node[fill=red!50!] (5) at (2,-4){5};
                    \end{scope}
                     \draw (1) edge node[above]{1} (2)
                       (2) edge node[right]{2} (3)
                       (1) edge node[right]{3} (3)
                       (1) edge node[right]{4} (4)
                       (4) edge node[above]{5} (3)
                       (3) edge node[right]{6} (5)
                       (4) edge node[right]{7} (5)
                       ;
            \end{tikzpicture}
        \end{minipage}
        \hfill
    % décrire problème
    \begin{minipage}[c]{0.45\linewidth}
        \begin{equation*}
            Sommets \left\{
        \begin{array}{cccccccc}
        & p_1 & p_2 & p_3 & p_4 & p_5 && y\\
        x_1 & 1 & 0 & 0 & 0 & 0 & & 1\\
        x_2 & 0 & 1 & 0 & 0 & 0 & & 1\\
        x_3 & 0 & 0 & 1 & 0 & 0 & & 1\\
        x_4 & 0 & 0 & 0 & 1 & 0 & & 1\\
        x_5 & 0 & 0 & 0 & 0 & 1 & & 1
        \end{array} \right.
        \end{equation*}
        \begin{equation*}
        ~~~~Ar\hat{e}tes \left\{
        \begin{array}{cccccccc}
        & p_1 & p_2 & p_3 & p_4 & p_5 && y\\
        x_1 & 1 & 1 & 0 & 0 & 0 & & 0\\
        x_2 & 0 & 1 & 1 & 0 & 0 & & 0\\
        x_3 & 1 & 0 & 1 & 0 & 0 & & 0\\
        x_4 & 1 & 0 & 0 & 1 & 0 & & 0\\
        x_5 & 0 & 0 & 1 & 1 & 0 & & 0\\
        x_6 & 1 & 0 & 1 & 0 & 1 & & 0\\
        x_7 & 1 & 0 & 0 & 1 & 1 & & 0\\
        \end{array} \right.
        \end{equation*}
    \end{minipage}

    Prouvons que pour tout graphe $G$ il est possible de construire un samlple
    $S = \left\{(x_i, y_i)\right\}$ tq $G$ est 3-coloriable ssi $\exists \varphi \in \mathcal{H}_3 \backslash \varphi \models S$ :

    \begin{itemize}
        \item Supposons que $G$ est 3-coloriable, construisons alors $\varphi$
        tq $\varphi \models S$ en illustrant sur l'exemple précédent.

        On construit d'abord une formule représentant la couleur rouge (ici les états 1 et 5) : $C_R = \neg p_2 \wedge \neg p_3 \wedge \neg p_4$. On procède de la même manière pour le
        vert et le bleu.

        $C_v = \neg p_1 \wedge \neg p_4 \wedge \neg p_5$

        $C_b = \neg p_1 \wedge \neg p_2 \wedge \neg p_4 \wedge \neg p_5$

        On obtient alors $\varphi = C_r \vee C_v \vee C_b$. \`A partir d'un
        graphe 3-colorié, on peut donc toujours construire $\varphi$ tq
        $\varphi \models S$.

        \item Supposons maintenant qu'on dispose d'une formule $\varphi = C_r \vee C_v \vee C_b$ tq $\varphi \models S$. Les sommets qui satisfont respectivement $C_r$,
        $C_v$ ou $C_b$ donnent la 3-coloration de $G$.

    \end{itemize}








\end{document}
