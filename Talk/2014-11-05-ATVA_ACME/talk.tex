\documentclass[svgnames]{beamer}
\mode<presentation>
\usefonttheme{serif}
\usecolortheme{dove}
\useinnertheme{rounded}
%\useoutertheme{smoothbars}
\setbeamercolor{item projected}{fg=black}
\setbeamertemplate{navigation symbols}{}

\usepackage[english]{babel}
\usepackage[latin1]{inputenc}
\usepackage{times}
\usepackage{amsthm,amssymb,amsmath,graphicx}
\usepackage{color}
\usepackage{gastex}
\usepackage{framed}
\usepackage{graphicx}
\usepackage{multicol}
\usepackage{ulem}
\usepackage{ifthen}
\usepackage{tikz}

%%%%%%%%%%%%%%%%%%%%%%%%%%%%%%%%%%%%%%%%%%%%%%%%%%%%%%%%%%%%%%%%%%%%%%%%%%%%%%%%%%%%%%%%
%%%%%%%%%%%%%%%%%%%% A non-original creation by Nathanaël Fijalkow and Victor Marsault %

\setbeamertemplate{frametitle}{
  \vskip-2pt
  \begin{beamercolorbox}[rightskip=2cm,leftskip=1em,dp=1ex,wd=12.8cm]{frametitle}
    \vskip2pt
    \usebeamercolor{frametitle}
    \begin{tikzpicture}
      \useasboundingbox (0,0) rectangle (0,0); 
      \ifthenelse{\insertframenumber<\inserttotalframenumber}
      { 
        \pgfmathsetmacro{\aimangle}{90-(\insertframenumber*360/\inserttotalframenumber)}
        \fill [fill=frametitle.fg,thin, color=gray!50,draw=black] (11.8,.2) -- (11.8,.6) arc (90:\aimangle:0.4) -- cycle;

      }{ 
        \fill[fill=frametitle.fg,thin, color=gray!50,draw=black] (11.8,0.2) circle (.4);
      }
      \fill[fill=frametitle.fg,thin, color=white,draw=black] (11.8,0.2) circle (.3);
      \node at (11.8, .2) [black,circle]{\normalsize\insertframenumber};
    \end{tikzpicture}
    \insertframetitle
    \vskip2pt
  \end{beamercolorbox}
}
%%%%%%%%%%%%%%%%%%%%%%%%%%%%%%%%%%%%%%%%%%%%%%%%%%%%%%%%%%%%%%%%%%%%%%%%%%%%%%%%%%%%%%%

\setbeamertemplate{blocks}[rounded]
\setbeamercolor{block title}{bg=normal text.bg!90!black}
\setbeamercolor{block body}{bg=normal text.bg!95!black}

\newcommand{\A}{\mathcal{A}}
\newcommand{\Q}{\mathbb{Q}}
\newcommand{\N}{\mathbb{N}}
\newcommand{\tr}[1]{\langle #1 \rangle}
\newcommand{\prob}[1]{\mathbb{P}_{#1}}
\newcommand{\set}[1]{\{ #1 \}}
\newcommand{\val}[1]{\text{val}(#1)}

  
\title{ACME: Automata with Counters,\\ Monoids and Equivalence}
\author{\textbf{Nathana\"el Fijalkow}, Denis Kuperberg}
\institute{LIAFA, Universit\'e Denis Diderot - Paris 7, France\\
Institute of Informatics, Warsaw University, Poland}

\date{November 5th, 2014}

\AtBeginSection[]
{
\addtocounter{framenumber}{-1}
  \begin{frame}<beamer>{Outline}
    \tableofcontents[currentsection]
  \end{frame}
}

\AtBeginSubsection[]
{
\addtocounter{framenumber}{-1}
  \begin{frame}<beamer>{Outline}
    \tableofcontents[currentsection,currentsubsection]
  \end{frame}
}

\begin{document}

\addtocounter{framenumber}{-1}

\begin{frame}
  \titlepage
\end{frame}

\begin{frame}{An algorithmic back-end: Stabilization Monoids}
\textbf{What?}
\begin{itemize}
	\item An algebraic structure with two operations: a binary composition and a unary operator $\sharp$,
	\item Generalizes the transition monoid of a non-deterministic automaton to two weighted settings.
\end{itemize}

\textbf{Where? When?}
\begin{itemize}
	\item First appeared in the Theory of Regular Cost Functions (Colcombet 2009),
	\item Later used for Probabilistic Automata (F., Gimbert, Oualhadj 2012).
\end{itemize}
\end{frame}

\begin{frame}{Non-deterministic automata}
\begin{figure}
\begin{center}
\begin{picture}(60,14)(0,0)
	\gasset{Nw=6,Nh=6,loopdiam=6}

  	\node[Nmarks=i,iangle=-90](0)(20,5){$0$}
  	\node(1)(40,5){$1$}
  	\node[Nmarks=r](F)(60,5){$F$}
  	\node(bottom)(0,5){$\perp$}

	\drawloop(0){$a$}
	\drawloop(1){$a$}
	\drawloop[loopangle=0](F){$a,b$}
	\drawloop[loopangle=180](bottom){$a,b$}

  	\drawedge(0,bottom){$b$}
  	\drawedge[curvedepth=2](0,1){$a$}
  	\drawedge[curvedepth=2](1,0){$b$}
  	\drawedge(1,F){$b$}
\end{picture}
\end{center}
\end{figure}
\vspace*{1em}

$$\tr{a} = 
\left(\begin{array}{cccc}
1 & 0 & 0 & 0 \\
0 & 1 & 1 & 0 \\
0 & 0 & 1 & 0 \\
0 & 0 & 0 & 1
\end{array}\right)
\qquad
\tr{b} = 
\left(\begin{array}{cccc}
1 & 0 & 0 & 0 \\
1 & 0 & 0 & 0 \\
0 & 1 & 0 & 1 \\
0 & 0 & 0 & 1
\end{array}\right)$$

\vspace*{1em}

$$I \cdot \tr{u} \cdot F = 1 \quad \textrm{ if and only if } \quad u \textrm{ is accepted.} $$
\end{frame}

\begin{frame}{Probabilistic automata}

\begin{figure}
\begin{center}
\begin{picture}(60,14)(0,0)
	\gasset{Nw=6,Nh=6,loopdiam=6}

  	\node[Nmarks=i,iangle=-90](0)(20,5){$0$}
  	\node(1)(40,5){$1$}
  	\node[Nmarks=r](F)(60,5){$F$}
  	\node(bottom)(0,5){$\perp$}

	\drawloop(0){$a:0.5$}
	\drawloop(1){$a$}
	\drawloop[loopangle=0](F){$a,b$}
	\drawloop[loopangle=180](bottom){$a,b$}

  	\drawedge(0,bottom){$b$}
  	\drawedge[curvedepth=2](0,1){$a:0.5$}
  	\drawedge[curvedepth=2](1,0){$b:0.7$}
  	\drawedge(1,F){$b:0.3$}
\end{picture}
\end{center}
\end{figure}
\vspace*{1em}

$$\tr{a} = 
\left(\begin{array}{cccc}
1 & 0 & 0 & 0 \\
0 & 0.5 & 0.5 & 0 \\
0 & 0 & 1 & 0 \\
0 & 0 & 0 & 1
\end{array}\right)
\qquad
\tr{b} = 
\left(\begin{array}{cccc}
1 & 0 & 0 & 0 \\
1 & 0 & 0 & 0 \\
0 & 0.7 & 0 & 0.3 \\
0 & 0 & 0 & 1
\end{array}\right)$$

\vspace*{1em}

$$I \cdot \tr{u} \cdot F = \prob{\A}(u)$$
\end{frame}

\begin{frame}{$B$-automata}

\begin{figure}
\begin{center}
\begin{picture}(60,14)(0,0)
	\gasset{Nw=6,Nh=6,loopdiam=6}

  	\node[Nmarks=i,iangle=-90](0)(20,5){$0$}
  	\node(1)(40,5){$1$}
  	\node[Nmarks=r](F)(60,5){$F$}
  	\node(bottom)(0,5){$\perp$}

	\drawloop(0){$a:+1$}
	\drawloop(1){$a$}
	\drawloop[loopangle=0](F){$a,b$}
	\drawloop[loopangle=180](bottom){$a,b$}

  	\drawedge(0,bottom){$b:+1$}
  	\drawedge[curvedepth=2](0,1){$a:r$}
  	\drawedge[curvedepth=2](1,0){$b:r$}
  	\drawedge(1,F){$b$}
\end{picture}
\end{center}
\end{figure}
\vspace*{1em}

$$\tr{a} = 
\left(\begin{array}{cccc}
0 & \bot & \bot & \bot \\
\bot & 1 & r & \bot \\
\bot & \bot & 0 & \bot \\
\bot & \bot & \bot & 0
\end{array}\right)
\qquad
\tr{b} = 
\left(\begin{array}{cccc}
0 & \bot & \bot & \bot \\
1 & \bot & \bot & \bot \\
\bot & r & \bot & 0 \\
\bot & \bot & \bot & 0
\end{array}\right)$$

\vspace*{1em}

$$I \cdot \tr{u} \cdot F = \A(u)$$
\end{frame}

\begin{frame}{Computing in Infinite Semirings}

Consider either	the rational semiring $(\Q,+,\times)$ or the tropical semiring $(\N \cup \set{\infty},\min,+)$:
\begin{itemize}
	\item An automaton $\A$ is given by a matrix $\tr{a}$ for each letter $a \in A$,
	\item We would like to finitely represent $\set{\tr{u} \mid u \in A^*}$.
\end{itemize}

\pause
\vskip1em
So we abstract away the precise values and consider two operators:
\begin{itemize}
	\item a binary composition law: matrix multiplication,
	\item a stabilization unary operator $\sharp$.
\end{itemize}

\pause
\vskip1em
Intuitively, $\tr{u}^\sharp$ represents $\lim_n \tr{u^n}$.
\end{frame}

\begin{frame}{Stabilization Monoid of an Automaton}

\begin{definition}
The Stabilization Monoid of $\A$ is the closure of $\set{\tr{a} \mid a \in A}$ under both operators.
\end{definition}

\vskip1em
The Stabilization Monoid of $\A$ contains a lot of informations about $\A$!
\end{frame}

\begin{frame}{Using the Stabilization Monoid}
\begin{columns}[t]
\begin{column}{0.6\textwidth}
\textbf{$B$-Automata}

\vskip1em
\begin{center}
\begin{itemize}
	\item Decide whether a $B$-automaton is bounded,
	\item Decide whether two $B$-automata are equivalent.
\end{itemize}
\end{center}
\end{column}
\begin{column}{0.5\textwidth}
\textbf{Probabilistic Automata}

\vskip1em
\begin{itemize}
	\item Decide whether a probabilistic automaton has (probably) value $1$,
	\item Decide whether a probabilistic automaton is leaktight.
\end{itemize}
\end{column}
\end{columns}
\end{frame}

\begin{frame}{The end.}
Thank you for your attention!
\end{frame}

\end{document}
