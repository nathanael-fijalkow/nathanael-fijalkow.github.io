\documentclass[svgnames]{beamer}
\mode<presentation>
\usefonttheme{serif}
\usecolortheme{dove}
\useinnertheme{rounded}
\setbeamercolor{item projected}{fg=black}
\setbeamertemplate{navigation symbols}{}

\usepackage[english]{babel}
\usepackage[latin1]{inputenc}
\usepackage{times}
\usepackage{amsmath}
\usepackage{amsfonts}
\usepackage{amssymb}
\usepackage{amsthm}
\usepackage{graphics}
\usepackage{multicol}
\usepackage{framed}
\usepackage{ulem}
\usepackage{ifthen}
\usepackage{tikz}
\usepackage{gastex}
\usepackage{ulem}
\usepackage{booktabs}


\newcommand{\set}[1]{\{ #1 \}}
\newcommand{\A}{\mathcal{A}}
\newcommand{\B}{\mathcal{B}}
\newcommand{\Parity}{\mathrm{Parity}}
\newcommand{\EXPTIME}{\mathrm{EXPTIME}}

\renewcommand{\ULthickness}{1.2pt}

%%%%%%%%%%%%%%%%%%%%%%%%%%%%%%%%%%%%%%%%%%%%%%%%%%%%%%%%%%%%%%%%%%%%%%%%%%%%%%%
%%%%%%%%%%%%%%%%%%%% A non-original creation by Nathana�l Fijalkow and myself %

\setbeamertemplate{frametitle}{%
  \vskip-2pt%
  \begin{beamercolorbox}[rightskip=2cm,leftskip=1em,dp=1ex,wd=12.8cm]{frametitle}%
    \vskip2pt%
    \usebeamercolor{frametitle}%
    \begin{tikzpicture}[scale=1]%
      \useasboundingbox (0,0) rectangle (0,0); %(-1,-1) rectangle (1,1);%
      \ifthenelse{\insertframenumber<\inserttotalframenumber}%
      { % uncomplete tart

        \pgfmathsetmacro{\aimangle}{90-(\insertframenumber*360/\inserttotalframenumber)}
        \fill [fill=frametitle.fg,thin, color=gray!50,draw=black] (11.8,.2) -- (11.8,.6) arc (90:\aimangle:0.4) -- cycle;%

      }{ % the full tart
        \fill[fill=frametitle.fg,thin, color=gray!50,draw=black] (11.8,0.2) circle (.4);%
      }%
      \fill[fill=frametitle.fg,thin, color=white,draw=black] (11.8,0.2) circle (.3);%
      \node at (11.8, .2) [black,circle]{\normalsize\insertframenumber};

    \end{tikzpicture}
    \insertframetitle%
    \vskip2pt%
  \end{beamercolorbox}%
}
%%%%%%%%%%%%%%%%%%%%%%%%%%%%%%%%%%%%%%%%%%%%%%%%%%%%%%%%%%%%%%%%%%%%%%%%%%%%%%%


\setbeamertemplate{blocks}[rounded]%
\setbeamercolor{block title}{bg=normal text.bg!90!black}
\setbeamercolor{block body}{bg=normal text.bg!95!black}

\begin{document}

\addtocounter{framenumber}{-1}

\title{Emptiness Of Alternating Tree Automata\\
Using Games With Imperfect Information}
\author{\underline{Nathana\"el Fijalkow} \and Olivier Serre \and Sophie Pinchinat}
\institute{Institute of Informatics, Warsaw University -- Poland 
\and LIAFA, Universit\'e Paris Diderot, Paris 7 \& CNRS -- France}
\date{FSTTCS, December 13th, 2013}

\begin{frame}
\maketitle
\end{frame}

\section{Introduction}

\begin{frame}{Objectives}
This talk is about \textit{one} approach to deal with alternating tree automata.
\pause

\vskip2em
Two settings: ``classical'' and ``qualitative''.
\end{frame}


\begin{frame}{Infinite binary trees}
\begin{figure}[!ht]
\begin{center}
\begin{picture}(50,40)(0,0)
	\gasset{Nh=5,Nw=5,Nframe=n,fillcolor=LightGray}

	\node(v_0)(25,30){$a$}
	\node(v_1)(10,20){$b$}
	\node(v_2)(40,20){$a$}
	\node(v_3)(0,10){$b$}
	\node(v_4)(20,10){$a$}
	\node(v_5)(30,10){$b$}
	\node(v_6)(50,10){$b$}

	\drawedge(v_0,v_1){}
	\drawedge(v_0,v_2){}
	\drawedge(v_1,v_3){}
	\drawedge(v_1,v_4){}
	\drawedge(v_2,v_5){}
	\drawedge(v_2,v_6){}

	\put(-1,0){$\begin{huge}\vdots\end{huge}$}
	\put(19,0){$\begin{huge}\vdots\end{huge}$}
	\put(29,0){$\begin{huge}\vdots\end{huge}$}
	\put(49,0){$\begin{huge}\vdots\end{huge}$}
\end{picture}
\end{center}
\end{figure}
\end{frame}

\begin{frame}{Definition of alternating tree automata (1/2)}

$\A = (Q,Q_\exists,Q_\forall,q_0,\delta,\Parity)$
\vskip1em
Transition relation: $\delta \subseteq Q \times A \times Q \times Q$

\begin{figure}[!ht]
\begin{center}
\begin{picture}(70,40)(0,0)
    \drawpolygon[linecolor=Red,linewidth=0.5,arcradius=3](0,18)(25,40)(50,18)
    \drawpolygon[linecolor=LightBlue,linewidth=0.5,arcradius=3](-5,3)(10,28)(25,3)
    \drawpolygon[linecolor=Purple,linewidth=0.5,arcradius=3](25,3)(40,28)(55,3)

	\gasset{Nframe=n}

	\node(q_0)(25,35){$q$}
	\node(q_1)(10,20){$p$}
	\node(q_2)(40,20){$r$}
	\node(q_3)(0,5){$q$}
	\node(q_4)(20,5){$q$}
	\node(q_5)(30,5){$p$}
	\node(q_6)(50,5){$r$}

	\put(65,25){$(q,a,p,r) \in \delta$}
	\put(65,15){$(p,b,q,q) \in \delta$}
	\put(65,10){$(r,a,p,r) \in \delta$}

	\gasset{Nh=5,Nw=5,Nframe=n,fillcolor=LightGray}

	\node(v_0)(25,30){$a$}
	\node(v_1)(10,15){$b$}
	\node(v_2)(40,15){$a$}
	\node(v_3)(0,0){$b$}
	\node(v_4)(20,0){$a$}
	\node(v_5)(30,0){$b$}
	\node(v_6)(50,0){$b$}

	\drawedge(v_0,v_1){}
	\drawedge(v_0,v_2){}
	\drawedge(v_1,v_3){}
	\drawedge(v_1,v_4){}
	\drawedge(v_2,v_5){}
	\drawedge(v_2,v_6){}
\end{picture}
\end{center}
\end{figure}
\end{frame}

\begin{frame}{Definition of alternating tree automata (2/2)}

Fix $\A$ alternating automaton, $t$ tree.
\vskip1em
$\looparrowright$ Two-player where Eve tries to show that the tree is accepted:
\begin{itemize}
	\item Eve picks the transitions from $Q_\exists$,
	\item Adam picks the transitions from $Q_\forall$,
	\item Adam also chooses directions.
\end{itemize}

\pause
\vskip1em
\textit{Classical} versus \textit{Qualitative} semantics:
\begin{itemize}
	\item $\A$ accepts $t$ if Eve has a strategy such that\\
	\textbf{all plays} consistent with the strategy are winning.
	\item $\A$ qualitatively accepts $t$ if Eve has a strategy such that\\
	\textbf{almost all plays} consistent with the strategy are winning.
\end{itemize}
\end{frame}

\begin{frame}{An example}
\begin{center}For all nodes labelled $a$, there is a branch without $b$'s\end{center}
\begin{figure}[!ht]
\begin{center}
\begin{picture}(120,40)(0,0)
	\gasset{Nh=5,Nw=5,Nframe=n,fillcolor=LightGray}

	\node(v_0)(25,30){$a$}
	\node(v_1)(10,20){$b$}
	\node(v_2)(40,20){$a$}
	\node(v_3)(0,10){$b$}
	\node(v_4)(20,10){$a$}
	\node(v_5)(30,10){$b$}
	\node(v_6)(50,10){$b$}

	\drawedge(v_0,v_1){}
	\drawedge(v_0,v_2){}
	\drawedge(v_1,v_3){}
	\drawedge(v_1,v_4){}
	\drawedge(v_2,v_5){}
	\drawedge(v_2,v_6){}

	\put(-1,0){$\begin{huge}\vdots\end{huge}$}
	\put(19,0){$\begin{huge}\vdots\end{huge}$}
	\put(29,0){$\begin{huge}\vdots\end{huge}$}
	\put(49,0){$\begin{huge}\vdots\end{huge}$}

	\put(100,60){$q \xrightarrow{a} p,q$}
	\put(100,40){$q \xrightarrow{a} p,q$}
	\put(100,20){}
	\put(100,0){$\begin{huge}\vdots\end{huge}$}

\end{picture}
\end{center}
\end{figure}
\end{frame}

\begin{frame}{Classical versus Qualitative semantics}

\begin{itemize}
	\item $L(\A)$: set of trees accepted by $\A$.
	\item $L^{=1}(\A)$: set of trees qualitatively accepted by $\A$.
\end{itemize}

\pause
\vskip2em

\begin{itemize}
	\item $\set{L(\A) \mid \A \textrm{ parity alternating automaton}}$: class of regular languages.
	\item $\set{L^{=1}(\A) \mid \A \textrm{ parity non-deterministic automaton}}$: class of qualitative languages,
	introduced by Carayol, Haddad and Serre.
\end{itemize}
\end{frame}

\begin{frame}{Properties of non-deterministic automata}
Regular languages and qualitative languages are incomparable.

\begin{center}
\begin{minipage}{0.4\textwidth}
\begin{center}\textbf{ Regular Languages}\end{center}
	\begin{itemize}
		\item Emptiness is decidable (no PTIME algorithm known),
		\item Closure under \textit{union}, \textit{intersection}, and \textit{projection},
		\item Closure under \textit{complementation}.
	\end{itemize}
\end{minipage}
\hspace{0.5cm}
\begin{minipage}{0.4\textwidth}
\begin{center}\textbf{ Qualitative Languages}\end{center}
	\begin{itemize}
		\item Emptiness is decidable in PTIME,
		\item Closure under \textit{union}, \textit{intersection}, and \textit{projection},
		\item \sout{Closure under \textit{complementation}}.
	\end{itemize}
\vspace{.75em}
\end{minipage}
\end{center}
\end{frame}

\begin{frame}{Properties of alternating automata}
\begin{itemize}
	\item[] \textbf{Input:} An alternating automaton $\A$
	\item[] \textbf{Output:} Is $L(\A)$ empty?
\end{itemize}

\vskip1em
\pause
\begin{theorem}[Muller and Schupp]
The emptiness problem for alternating parity automata with the classical semantics is
$\EXPTIME$-complete.
\end{theorem}

\begin{theorem}[Our paper]
\begin{itemize}
	\item The emptiness problem for alternating B\"uchi automata with the qualitative semantics is
$\EXPTIME$-complete.
	\item The emptiness problem for alternating CoB\"uchi automata with the qualitative semantics is
undecidable.
\end{itemize}
\end{theorem}

\pause
We use a different technique, through games of imperfect information.
\end{frame}

\section{A solution through the simulation technique (Muller and Schupp)}

\addtocounter{framenumber}{-1}
\begin{frame}<beamer>{Outline}
    \tableofcontents[currentsection]
\end{frame}

\begin{frame}{A sketch of the solution}
Fix $\A$ alternating parity automaton.

\begin{enumerate}
	\item construct an equivalent non-deterministic automaton $\B$,
	\item solve the emptiness problem for $\B$.
\end{enumerate}

\vskip1em
\pause
The emptiness problem for non-deterministic automata
reduces to solving a two-player game,
where Eve tries to show that there exists an accepted tree:
\begin{itemize}
	\item Eve picks the tree $t$ \textit{and} the transitions,
	\item Adam only chooses directions.
\end{itemize}
\end{frame}

\begin{frame}{Questions}
\begin{itemize}
	\item Can we directly reduce the emptiness problem for alternating automata
	to solving a two-player game?
	\item How to handle the qualitative semantics, for which no simulation is known?
\end{itemize}
\end{frame}

\begin{frame}{Problem}
\begin{center}
\color{Red}{In the naively extended previous game,\\
Eve can cheat if she sees what Adam is doing!}
\end{center}

\vskip1em
\pause
Consider a universal word automaton where Adam can decide in the first move to:
\begin{enumerate}
	\item check that the word contains a $b$,
	\item or check that the word does not contain any $b$
\end{enumerate}
The language accepted is empty but Eve can wait to see whether Adam
chooses option $1$ or $2$ and pick the next letters to contradict Adam's choice.
\end{frame}

\section{A different technique through games of imperfect information}

\addtocounter{framenumber}{-1}
\begin{frame}<beamer>{Outline}
    \tableofcontents[currentsection]
\end{frame}

\begin{frame}{Games of imperfect information}

We directly reduce the emptiness problem for alternating automata
to solving a two-player imperfect-information game,
where Eve tries to show that there exists an accepted tree:
\begin{itemize}
	\item Eve picks the tree $t$ \textit{and} a positional strategy for her transitions,
	\item Adam chooses transitions and directions.
	\item \color{Red}{Eve does not see what Adam does!}
\end{itemize}

\vskip1em
\pause
The new key technical ingredient is 
a positional determinacy theorem for infinite (chronological) arenas:

\begin{theorem}
If Eve has an almost surely winning strategy for B\"uchi,\\
then she has a positional one.
\end{theorem}
\end{frame}

\begin{frame}{Summary}
\begin{theorem}[Our paper]
\begin{itemize}
	\item The emptiness problem for alternating B\"uchi automata with the qualitative semantics is
$\EXPTIME$-complete.
	\item The emptiness problem for alternating CoB\"uchi automata with the qualitative semantics is
undecidable.
\end{itemize}
\end{theorem}
\vskip1em
\pause
Remark: the second item shows that there is no simulation theorem for qualitative automata!
\end{frame}

\begin{frame}{A generic result}
To solve the emptiness problem of alternating automata with a semantics $Acc$,
one can reduce it to a two-player imperfect-information game, and:
\begin{enumerate}
	\item to prove the construction correctness,
	show a positional determinacy result for $Acc$,
	\item solve the obtained imperfect-information game with winning condition $Acc$.
\end{enumerate}

\vskip2em
\pause
Thank you!
\end{frame}

\end{document}